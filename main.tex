\documentclass[11pt]{article}
\usepackage{amsmath, amssymb, amscd, amsthm, amsfonts}
\usepackage{graphicx}
\usepackage{hyperref}
\hypersetup{
    colorlinks=true,
    linkcolor=blue,
    filecolor=magenta,      
    urlcolor=cyan,
    pdftitle={Overleaf Example},
    pdfpagemode=FullScreen,
    }
\usepackage[dvipsnames]{xcolor}

\oddsidemargin 0pt
\evensidemargin 0pt
\marginparwidth 40pt
\marginparsep 10pt
\topmargin -20pt
\headsep 10pt
\textheight 8.7in
\textwidth 6.65in
\linespread{1.2}

\title{Course Report: Classification}
\author{Nan He}
\date{}

\newtheorem{theorem}{Theorem}
\newtheorem{lemma}[theorem]{Lemma}
\newtheorem{conjecture}[theorem]{Conjecture}

\newcommand{\rr}{\mathbb{R}}

\newcommand{\al}{\alpha}
\DeclareMathOperator{\conv}{conv}
\DeclareMathOperator{\aff}{aff}

\begin{document}

\maketitle

%\begin{abstract}
%(Abs)
%\end{abstract}

\section{Main objective of the analysis}\label{section-introduction-1}
Acknowledgement: this is a course project for \href{https://www.coursera.org/professional-certificates/ibm-machine-learning}{IBM Machine Learning professional certificate}. The project notebook can be accessed \href{https://github.com/henankf223/Assignment-2/blob/26433c172a58e934f1039ef8de647232f05e747c/PES_DVR_pd.ipynb}{here}.

\section{Description of the data set}\label{section-introduction-2}

\section{Data exploration, cleaning, and feature engineering}\label{section-introduction-3}

\section{Testing different models}\label{section-model}
(logistic regression, svm-gaussian, random-forest, Gradient-Boosting)

\section{Final model choices and analysis}\label{section-pred}
(plot the CM)

\section{Key Findings and Insights}\label{section-find}

\section{Summary and suggestions for next steps}\label{section-sugg}

%\bibliographystyle{alpha}
%\bibliography{references} % see references.bib for bibliography management

\end{document}
